\documentclass[,11pt,a4paper]{article}
\usepackage[utf8]{inputenc}
\usepackage[italian]{babel}
\usepackage{amsmath}
\usepackage{amsfonts}
\usepackage{float}
\usepackage{amssymb}
\usepackage{xcolor}
\usepackage{subfig}
\usepackage{microtype}
\usepackage{lmodern}
\usepackage{graphicx}
\usepackage{wrapfig}
\usepackage{bbold}
\usepackage{hyperref}
%\usepackage{fourier}
\hypersetup{colorlinks=true, linkcolor=black, urlcolor= black}
\usepackage{fancyhdr}
\usepackage{graphicx}
\usepackage{simplewick}
\usepackage{multimedia}
\usepackage{animate} 
\let\arrvec=\vec
\renewcommand\vec{\mathbf}
\newcommand{\0}{\vert 0 \rangle}
\newcommand{\1}{\vert 1 \rangle}

\usepackage{booktabs}
\usepackage[left=20mm,right=20mm,top=20mm,bottom=20mm]{geometry}
\title{Topology in Condended Matter Physics}

\author{Roberto Menta}

\begin{document}
\maketitle
\tableofcontents
\section{Barry phase}
La meccanica quantistica ci dice che  se conosco lo stato $\vert \psi \rangle$ di un sistema quantistico allora sono a conoscenza anche della sua evoluzione temporale $U(t) = e^{-i \mathcal{H} t / \hbar}$ dove $\mathcal{H}$ è l'Hamiltoniana che descrive la dinamica del sistema.  Possiamo quindi scrivere che 
\[
\vert \psi(\vec{r},t)  \rangle = e^{-i \mathcal{H} t / \hbar} \vert \psi(\vec{r}) \rangle\] 
L'operatore di evoluzione temporale non contribuisce alla densità di probabilità ossia al modulo quadro dello stato.  Di fatto è una fase detta \emph{fase dinamica}. \\ \\ 
Consideriamo una Hamiltoniana variabile temporalmente in modo adiabatico dipendente da $\vec{r}(t) = (r_1(t),r_2(t),\ldots)$ e in generale da una serie di parametri esterni (flusso magnetico, campo elettrico, etc...).  Consideriamo un cammino $\mathcal{C}$ nello spazio dei parametri $r_i(t)$ i quali vengono variati lentamente con il tempo.  Si introduce una base ortonormale \emph{istantanea} $\{\vert n(\vec{r}(t)) \rangle \}$ di autostati della Hamiltoniana $\mathcal{H}(\vec{r}(t))$ :
\[
\mathcal{H}(\vec{r}) \vert n(\vec{r}) \rangle = \varepsilon_n(\vec{r}) \vert n(\vec{r}) \rangle.
\]
Questa equazione determina gli stati di base $\{\vert n(\vec{r}(t)) \rangle \}$ a meno di un fattore di fase.  \'E noto infatti che un qualunque vettore nello spazio di Hilbert degli stati è detto \emph{raggio} poiché è definibile sempre a meno di un fattore $e^{i\alpha}$ ininfluente quando se ne fa il modulo quadro. Una possibile scelta per la fase è dettata dalla scelta di gauge.  Possiamo richiedere che la fase di ogni vettore di base sia differenziabile e monodroma lungo il cammino $\mathcal{C}$.  \\ \\
Ammettiamo che il sistema quantistico sia inizialmente preparato in uno stato puro $\vert n(\vec{r}(0)) \rangle$, autostato della Hamiltoniana istantanea $\mathcal{H}(\vec{r}(0))$.  Ora, variamo lentamente $\vec{r}(t)$ lungo il cammino $\mathcal{C}$,  lo stato iniziale evolverà con $\mathcal{H}(\vec{r})$ e instante per istante lo stato $\vert n(\vec{r}(t + dt)) \rangle$ rimarrà autostato della Hamiltoniana allo stesso tempo $t+dt$. Questo risultato deriva dal teorema adiabatico della meccanica quantistica.  La domanda che ci poniamo è: cosa succede alla fase? Ridefiniamo lo stato di base come 
\[\vert \psi_n(t) \rangle = e^{-i \theta(t)} \vert n(\vec{r}(t)) \rangle\]
La fase $\theta(t)$ non può far zero in quanto oltre all'ipotesi dell'esistenza di una ulteriore fase \emph{geometrica}, deve contenere il contributo all'energia a cui è associato lo stato. Consideriamo l'evoluzione dinamica del sistema dettata dell'equazione di Schr\"odinger,
\[
\mathcal{H}(\vec{r}) \vert \psi_n(t) \rangle = i \hbar \dfrac{\partial}{\partial t} \vert \psi_n(t) \rangle
\] che in termini di $\vert n(\vec{r}) \rangle$ :
\[
\varepsilon_n(\vec{r}(t)) \vert n(\vec{r}(t)) \rangle = \left( \hbar \dot{ \theta}(t) + i \hbar  \dfrac{\partial}{\partial t} \right)  \vert n(\vec{r}(t)) \rangle .
\]
Prendiamo adesso il prodotto scalare con il bra $\langle n(\vec{r}(t)) \vert$ e assumiamo che il vettore di stato sia normalizzato ad $1$. Si ottiene 
\[\
\varepsilon_n(\vec{r}(t)) = \hbar \dot{\theta}(t) + \hbar i \langle n(\vec{r}(t)) \vert \partial_t \vert n(\vec{r}(t)) \rangle
\]
da cui, risolvendo per $\theta(t)$ si ottiene 
\[
\theta(t) = \dfrac{1}{\hbar} \int_0^t \varepsilon_n(\vec{r}(t')) dt' - i \int_0^t \langle n(\vec{r}(t')) \vert \partial_{t'} \vert n(\vec{r}(t')) \rangle dt'
\]
dove il primo contributo è la conosciuta \emph{fase dinamica} mentre (meno) il secondo contributo è la \emph{fase di Berry}.  In maniere più compatta abbiamo ottenuto che 
\[
\vert \psi_n(t) \rangle = \exp\left[-\dfrac{i}{\hbar} \int_0^t \varepsilon_n(\vec{r}(t')) dt' \right] \exp(i\gamma_n) \vert n(\vec{r}(t)) \rangle
\] 
dove $\gamma_n$ è la \emph{Berry phase},
\[
\gamma_n = i \int_0^t \langle n(\vec{r}(t')) \vert \partial_{t'} \vert n(\vec{r}(t')) \rangle dt'
\]
Questo contributo di fase discende dal fatto che gli stati al tempo $t$ e al tempo $t+dt$ non sono identici. Possiamo rimuovere la dipendenza dal tempo dall'espressione precedente.  Consideriamo il cammino nello spazio dei parametri $\mathcal{C}$,  la fine del cammino corrisponde al tempo $\tilde{t}$. Si ha che 
\[
\gamma_n = i \int_0^{\tilde{t}} \langle n(\vec{r}(t')) \vert \nabla_{\vec{r}} \vert n(\vec{r}(t')) \rangle \dfrac{d\vec{r}}{dt'} dt'\] da cui
\[
\boxed{
\gamma_n = i \int_{\mathcal{C}} \langle n(\vec{r}) \vert \nabla_{\vec{r}} \vert n(\vec{r})\rangle d\vec{r}
}
\]
In analogia con il trasporto elettrico in un campo elettromagnetico possiamo definire la \emph{connessione di Berry} (detto anche potenziale vettore di Berry),
\[
\vec{A}_n(\vec{r}) = i \langle n(\vec{r}) \vert \nabla_{\vec{r}} \vert n(\vec{r})\rangle, \ \ \ \ \ \ \ \ \ \ \ \ \gamma_n = \int_{\mathcal{C}} \vec{A}_n(\vec{r}) d\vec{r}
\]
Il significato della parola ''connessione" è legato al concetto di trasporto. Infatti ci dà informazione su come un oggetto cambia quando è trasportato da una parte all'altra della nostra varietà. Un esempio di connessione è il trasporto parallelo che ci dice come trasportare un vettore che vive in una superficie ad un altro punto rimanendo parallelo.
La fase di Berry è reale infatti 
\[
\gamma_n^* = -i \int_{\mathcal{C}} -  \langle n(\vec{r}) \vert \nabla_{\vec{r}} \vert n(\vec{r})\rangle d\vec{r} = \gamma_n
\] 
avendo usato che $ \langle n(\vec{r}) \vert \nabla_{\vec{r}} \vert n(\vec{r})\rangle = -  \langle n(\vec{r}) \vert \nabla_{\vec{r}} \vert n(\vec{r})\rangle^*$.  La $i$ davanti all'integrale rende reale la fase di Berry in quanto anche l'argomento dell'integrale è immaginario. Il fatto che $\gamma_n \in \mathbb{R}$ significa che la fase di Berry non caratterizza un decadimento.
\[
\gamma_n = - \Im \int_{\mathcal{C}} \langle n(\vec{r}) \vert \nabla_{\vec{r}} \vert n(\vec{r})\rangle d\vec{r}.
\]
La connessione di Berry $\vec{A}_n(t)$ è dipendente dalla scelta di gauge che facciamo.  Consideriamo, per esempio, la trasformazione $\vert n(\vec{r}) \rangle \to e^{i \Lambda(\vec{r})} \vert n(\vec{r}) \rangle$ dove $\Lambda(\vec{r})$ è una funzione differenziabile infinite volte e a singolo valore.  Allora, il potenziale vettore trasforma come ($\langle n(\vec{r}) \vert \to e^{-i \Lambda(\vec{r})} \langle n(\vec{r}) \vert$) : 
\[
\begin{array}{rlc}
\vec{A}_n(\vec{r}) & \to i e^{-i \Lambda(\vec{r})} \left( \langle n(\vec{r}) \vert i \nabla_{\vec{r}} \Lambda(\vec{r}) e^{i \Lambda(\vec{r})} \vert n(\vec{r})\rangle + \langle n(\vec{r}) \vert e^{i \Lambda(\vec{r})} \nabla_{\vec{r}} \vert n(\vec{r})\rangle \right) \\ \\ &= \vec{A}_n(\vec{r}) - \nabla_{\vec{r}} \Lambda(\vec{r})
\end{array}
\]
Ossia trasforma come ci si aspetta per trasformazioni di gauge.  Una conseguenza della buona trasformabilità sotto il gruppo di gauge è che la fase di Berry cambia in un periodo $\tilde{t}$ infatti
\[
\Delta \gamma_n = - \int_{\mathcal{C}} \nabla_{\vec{r}} \Lambda(\vec{r}) d\vec{r} = \Lambda(\vec{r}(0)) - \Lambda(\vec{r}(\tilde{t})).
\]
Dopo la rivoluzionaria scoperta di Berry, molti fisici pensavano che attraverso una scelta appropriata di gauge sarebbe stato possibile eliminare la fase $\gamma_n$.  In realtà non è così. La fase di Berry ha una natura geometrico-fondamentale e non è eliminabile per scelta di gauge. \\ \\ 
Consideriamo un cammino chiuso $\mathcal{K}$, ciò significa che $\vec{r}(0) = \vec{r}(\tilde{t})$. Questo significa che dopo un tempo pari al periodo $\tilde{t}$, la base di autostati della Hamiltoniana $\mathcal{H}$ torna ad essere la stessa ossia $\vert n(\vec{r}(0)) \rangle = \vert n(\vec{r}(\tilde{t})) \rangle$. Quest'ultima condizione è invariante sotto trasformazioni di gauge : 
\[
e^{i\Lambda(\vec{r}(0))}\vert n(\vec{r}(0)) \rangle = e^{i\Lambda(\vec{r}(\tilde{t}))}\vert n(\vec{r}(\tilde{t})) \rangle = e^{i\Lambda(\vec{r}(0))}\vert n(\vec{r}(\tilde{t})) \rangle 
\]
\
\[
\ \ \ \ \ \Rightarrow \ \ \ \ \ \boxed{\Lambda(\vec{r}(\tilde{t})) - \Lambda(\vec{r}(0)) = 2\pi m \ , \ \ \ \ \ \ m \in \mathbb{Z}}
\]
La fase di Berry non può essere cancellata in una evoluzione adiabatica lungo un percorso $\mathcal{K}$ chiuso nello spazio dei parametri ! L'unico modo per cancellare la fase è che la differenza sia multiplo intero di $2\pi$.  Dunque,  in generale, la fase di Berry è eliminabile se consideriamo cammini $\mathcal{C}$ non chiusi, è invece di natura fondamentale, \emph{invariante di gauge} e indipendente dalla dipendenza temporale dei parametri $\vec{r}(t)$,  se consideriamo cammini chiusi $\mathcal{K}$.  
\[
\gamma_n = \oint_{\mathcal{K}} \vec{A}_n(\vec{r}) d\vec{r} 
\]
In quest'ultimo caso possiamo utilizzare il teorema di Stokes :
\begin{equation}\label{phase_derivate}
\begin{array}{rlc}
\gamma_n &= \displaystyle - \Im \oint_{\mathcal{K}} d\vec{r} \langle n(\vec{r}) \vert \nabla \vert n(\vec{r})\rangle  =  - \Im \oint_{\mathcal{S}} d\vec{S}  \cdot \left( \nabla \wedge \langle n(\vec{r}) \vert \nabla \vert n(\vec{r})\rangle \right) \\ \\ 
&= \displaystyle - \Im \oint_{\mathcal{S}} dS_i \epsilon^{ijk}  \nabla_j  \langle n(\vec{r}) \vert \nabla_k \vert n(\vec{r})\rangle \\ \\ 
&= \displaystyle - \Im \oint_{\mathcal{S}} d\vec{S}  \cdot \left(  \langle \nabla n(\vec{r}) \vert \wedge \vert \nabla n(\vec{r})\rangle \right)
\end{array}
\end{equation}
dove $\mathcal{K} = \partial \mathcal{S}$ e $\langle \nabla n(\vec{r}) \vert \wedge \vert \nabla n(\vec{r})\rangle$ è la \emph{curvatura di Berry}. In particolare 
\[
F_{jk} = \langle \nabla_j n(\vec{r}) \vert \nabla_k n(\vec{r})\rangle - \langle \nabla_k n(\vec{r}) \vert \nabla_j n(\vec{r})\rangle
\]
è la curvatura di Berry, in analogia con il campo magnetico (rotore del potenziale vettore). Se rendiamo l'analogia ancora più forte con l'elettrodinamica  possiamo riscrivere la curvatura come un tensore di campo di gauge 
\[
\Omega_{\mu\nu} = \dfrac{\partial}{\partial r^{\mu}}A_{\nu}(\vec{r}) -  \dfrac{\partial}{\partial r^{\nu}}A_{\mu}(\vec{r}) 
\]

\section{Computation of Berry phase}
Calcolare numericamente la fase di Berry è molto complesso a causa della presenza delle derivate degli autostati. Se procedessimo nella diagonalizzazione della Hamiltoniana, otterremmo una base di autostati per ogni $\vec{r}$.  La procedura di diagonalizzazione produrrebbe austostati con fattori di fase molto diversi tra loro, impedendo così di prendere le derivate.  Dovremmo per prima cosa rendere la fase di Berry \emph{gauge-smooth}, ma questa procedura non è affatto banale. \\ \\
Ricaviamo una formula che si invariante di gauge a vista.  Siano $\{ \vert m(\vec{r}) \rangle \}$ un set completo di autostati esatti della Hamiltoniana istantanea
(ossia $\sum_m \vert m(\vec{r}) \rangle \langle m(\vec{r}) \vert = 1 \ \ \ \forall \ \vec{r}$), allora 
\[
\begin{array}{rlc}
\mathcal{F}_{jk}(\vec{r}) &= \epsilon_{ijk} \langle \nabla_j n(\vec{r}) \vert \nabla_k n(\vec{r}) \rangle \\ \\
&= \epsilon_{ijk} \langle \nabla_j n(\vec{r})\vert n \rangle \langle n \vert \nabla_k n(\vec{r}) \rangle + \displaystyle  \sum_{m \neq n}  \epsilon_{ijk} \langle \nabla_j n(\vec{r})\vert m \rangle \langle m \vert \nabla_k n(\vec{r}) \rangle \\ \\
&=  \langle \nabla_j n(\vec{r})\vert n \rangle \langle n \vert \nabla_k n(\vec{r}) \rangle + \displaystyle  \sum_{m \neq n} \langle \nabla_j n(\vec{r})\vert m \rangle \langle m \vert \nabla_k n(\vec{r}) \rangle \ + \\ \\ 
& \ - \langle \nabla_k n(\vec{r})\vert n \rangle \langle n \vert \nabla_j n(\vec{r}) \rangle - \displaystyle  \sum_{m \neq n} \langle \nabla_k n(\vec{r})\vert m \rangle \langle m \vert \nabla_j n(\vec{r}) \rangle
\end{array}
\]
Possiamo eliminare il primo termine perché sia $ \langle \nabla_j n(\vec{r})\vert n \rangle$ che $\langle n \vert \nabla_k n(\vec{r}) \rangle $ sono immaginari mentre la fase di Berry, come  visto precedentemente è reale. Dunque non contribuiscono al calcolo della fase di Berry. Si ha quindi che 
\[
\gamma_n = - \Im \int \int_{\mathcal{S}} dS_i \sum_{m \neq n}  \epsilon_{ijk} \langle \nabla_j n(\vec{r})\vert m \rangle \langle m \vert \nabla_k n(\vec{r}) \rangle
\]
Riscriviamo ora $\langle m \vert \nabla n \rangle$ e conseguentemente anche $ \langle \nabla n \vert m \rangle$.  Si sa che $\langle n \vert m \rangle = 0 \ \ \forall \ n \neq m$ e che $\vert n \rangle, \vert m \rangle$ sono autostati della Hamiltoniana $\mathcal{H}/\vec{r})$ istantanea.  Allora 
\[
\varepsilon_n \langle m \vert \nabla n \rangle = \langle m \vert \nabla (\mathcal{H} n )\rangle = \langle m \vert \nabla \mathcal{H} \vert n \rangle + \varepsilon_m \langle m \vert \nabla n \rangle
\]
da cui 
\[
\langle m \vert \nabla n \rangle = \dfrac{\langle m \vert \nabla \mathcal{H} \vert n \rangle}{\varepsilon_n - \varepsilon_m}
\]
e similmente 
\[
\langle \nabla n \vert m \rangle = \dfrac{\langle n \vert \nabla \mathcal{H} \vert m \rangle}{\varepsilon_n - \varepsilon_m}
\]
Sostituendo nell'espressione per $\gamma_n$ si trova che
\begin{equation}\label{phase_derivite_out}
\boxed{
\gamma_n = - \Im \int \int_{\mathcal{S}} dS_i \sum_{m \neq n}  \epsilon_{ijk} \dfrac{\langle n(\vec{r}) \vert \nabla_j \mathcal{H}(\vec{r}) \vert m(\vec{r}) \rangle\langle m(\vec{r}) \vert \nabla_k \mathcal{H}(\vec{r}) \vert n(\vec{r}) \rangle}{(\varepsilon_n(\vec{r}) - \varepsilon_m(\vec{r}))^2}
}
\end{equation}
Questa formula è manifestamente invariante di gauge ed ha il vantaggio di non avere dipendenza dalla derivate degli autostati e dalle loro fasi. Può essere quindi calcolata per ogni scelta di gauge.  Ricapitolando, siamo partita dalla \eqref{phase_derivate} e siamo arrivati alla \eqref{phase_derivite_out} in cui la derivata sui parametri $\vec{r}$ è stata scaricata sulla Hamiltoniana istantanea. Possiamo interpretare la curvatura di Berry come il risultato dell'interazione tra gli stati adiabatici $n$ e gli autostati esatti $m$.
\end{document}
